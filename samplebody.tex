\section{Fonts}

\topic{Features.}
Change the font of \texttt{texttt}.

\section{Paper or tech report}

\topic{Features.}
One could maintain the conference paper and technical reports (full version) is
the same set of files.

\begin{itemize}
\item This is a normal sentence.
\item Only paper: \onlypaper{This sentence only appears in the paper.}
\item Only tech: \onlytech{This sentence only appears in the technical report.}
\end{itemize}

\section{Colors}\label{sec:color}

\topic{Features.}
Manually set colors of hyper links.

\begin{itemize}
\item Links (medium-red): \Cref{sec:color}
\item Citation (medium-blue): \cite{lin2017boosting}
\item URL (medium-blue): \url{google.com}
\end{itemize}

\section{Algorithm}

\topic{Features.}
(1) Comments are in dark blue. (2) One may use a cut-off rule \texttt{algrule}
to cut a piece of long pseudo-code.

\begin{algorithm}
  \SetKwFunction{MyFunc}{MyFunc}%
  \KwIn{inputs}%
  \KwOut{outputs}%
  $X\leftarrow$ \MyFunc{} \tcp*{Hola!} \label{algo:l1}%
  $X\leftarrow$ \MyFunc{}\;%
  $X\leftarrow$ \MyFunc{}\label{algo:l3}\;%
  \algrule %
  \myproc{\MyFunc{}}{%
    \KwRet{``Hello world''}\;%
  }
  \caption{Hello World} \label{algo:findcp}
\end{algorithm}

\section{Moving proofs}

The package \texttt{sty/moveproofs.sty} is adapted from repository
\url{https://github.com/thisisdhaas/moveproofs}.
The package is useful if one prefers writing theorems and their proofs in the
same LaTex files.
(I am too lazy to manually move proofs to appendix.)

\begin{itemize}
\item To show all ``makeproof'' proofs in the appendix, use:
\begin{verbatim}
\usepackage[location=appendix]{sty/moveproofs}
\end{verbatim}
\item To show all ``makeproof'' proofs in its original location, use:
\begin{verbatim}
\usepackage{sty/moveproofs}
\end{verbatim}
\end{itemize}

Here goes a sample theorem and its proof.
The proof appears in the appendix if \texttt{location=appendix}.

\begin{theorem}\label{thm:sample}
  A sample theorem.
\end{theorem}

\makeproof{thm:sample}{ %
  The proof of the sample theorem (\Cref{thm:sample}).
}

\begin{proof}
  A normal \texttt{proof} environment.
\end{proof}

\section{Restating theorems}

The package \texttt{thm-restate}
\footnote{\url{http://tug.ctan.org/macros/latex/exptl/thmtools/thmtools.pdf} }
is useful for re-stating theorems in the appendix (e.g., right before the proof
manually put in the appendix).

\begin{restatable}[Euclid]{theorem}{firsteuclid}
  \label{thm:euclid}%
  For every prime $p$, there is a prime $p'>p$.
  In particular, the list of primes,
  \begin{equation}\label{eq:1}
    2,3,5,7,\dots
  \end{equation}
  is infinite.
\end{restatable}

Use \verb|\firsteuclid*| to restate the theorem. 
The outputs are as follow.

\firsteuclid*

\section{Smart references}

The package \texttt{cleveref} is
useful\footnote{\url{http://mirrors.ctan.org/macros/latex/contrib/cleveref/cleveref.pdf}}.

Some example equations
\begin{align}
  &\log xy = \log x + \log y \label{eq:log}\\
  &a^2 +b^2=c^2\label{eq:pyth} \\
  &E=mc^2 \label{eq:emc2}
\end{align}

Examples of \texttt{cref} and \texttt{crefrange}:
\begin{itemize}
\item \verb|\Cref{eq:log}|: \Cref{eq:log}
\item \verb|\Crefrange{eq:log}{eq:emc2}|: \Crefrange{eq:log}{eq:emc2}
\item \verb|\cref{eq:log}|: \cref{eq:log}
\item \verb|\crefrange{eq:log}{eq:emc2}|: \crefrange{eq:log}{eq:emc2}
\item \verb|\Cref{algo:l1}|: \Cref{algo:l1}
\item \verb|\Crefrange{algo:l1}{algo:l3}|: \Crefrange{algo:l1}{algo:l3}
\item \verb|\cref{algo:l1}|: \cref{algo:l1}
\item \verb|\crefrange{algo:l1}{algo:l3}|: \crefrange{algo:l1}{algo:l3}
\end{itemize}

From the document of \texttt{cleveref}: ``To disable all use of abbreviations in
the default cross-reference names (e.g., for VLDB) , pass the \texttt{noabbrev}
option to the \texttt{cleveref} package.''

\section{More Commands}

More commands for adding notes to the paper.

\begin{itemize}
\item \verb|\yishi{hi}|: \yishi{hi}
\item \verb|\notes{hi}|: \notes{hi}
\item \verb|\todo{hi}|: \todo{hi}
\end{itemize}

\section{Squeezing the paper}

Search ``squeeze'' in \texttt{main.tex}!